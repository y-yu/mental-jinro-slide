\definecolor{links}{HTML}{2A1B81}
\hypersetup{colorlinks,linkcolor=,urlcolor=links}

\usetheme{Boadilla}
\usecolortheme{seahorse}
\usefonttheme{serif}
\beamertemplatenavigationsymbolsempty

\usepackage{luacode}
\usepackage{luatexja}
\usepackage{pgfpages}
\usepackage[osf]{mathpazo}
\usepackage{fontspec}

\begin{luacode*}
  USE_IPAFONT = os.getenv"USE_IPAFONT"
  
  if USE_IPAFONT == "true" then
    tex.sprint("\\AtBeginDocument{\\usepackage[ipaex, deluxe, expert]{luatexja-preset}}")
  else
    tex.sprint("\\AtBeginDocument{\\usepackage[hiragino-pro, deluxe, expert]{luatexja-preset}}")
    tex.sprint("\\AtBeginDocument{\\setmainjfont[BoldFont=Hiragino Kaku Gothic Pro W6]{Hiragino Kaku Gothic Pro W3}}")
  end
\end{luacode*}

\usepackage{epigraph}
\usepackage{etoolbox}
\usepackage{tikz}
\usepackage{framed}
\usepackage{libertine}
\usepackage[final]{listings}
\usepackage{amsmath}
\usepackage{mathtools}

%\setbeameroption{show notes on second screen=right}

%\usetikzlibrary{arrows,automata,shapes,backgrounds}

\setmainfont[Numbers=OldStyle, BoldFont=Palatino Bold]{Palatino}
\setsansfont{CMU Sans Serif}
\setmonofont{CMU Typewriter Text}

\title[Mental Jinro]{%
  \href{https://github.com/y-yu/mental-jinro-slide}{Mental Jinro} \\
  \href{http://connpass.com/event/30661/}{\normalsize tsukuba.pm \#3}
}
\author{吉村 優}
\date{May 14, 2016}
\institute[\url{https://twitter.com/\_yyu\_}]{%
  \url{https://twitter.com/\_yyu\_}\\
  \url{http://qiita.com/yyu}\\
  \url{https://github.com/y-yu}\\
}

\input{./lib/quotebox.tex}
\input{./lib/listings.tex}
\input{./lib/footnotemark.tex}
\input{./lib/mydescription.tex}
\input{./lib/ballon.tex}

\newcommand\ballref[1]{%
\tikz \node[circle, shade,ball color=structure.fg,inner sep=0pt,%
  text width=8pt,font=\tiny,align=center] {\color{white}\ref{#1}};
}

%\everymath{\displaystyle}

\begin{document}

\frame{\maketitle}

\section{自己紹介}
\begin{frame}[fragile]
  \frametitle{自己紹介}
  
  \begin{columns}
    \begin{column}{0.4\textwidth}
      \centering
      \begin{figure}
        \includegraphics[width=0.9\textwidth]{img/bird2x.png}
      \end{figure}
    \end{column}
    \begin{column}{0.6\textwidth}
      \begin{itemize}
        \item<2-> \href{https://www.coins.tsukuba.ac.jp/}{筑波大学 情報科学類} 学士(COINS11)
        \item<3-> \href{http://www.word-ac.net/}{WORD編集部} OB
        \item<4-> \href{http://logic.cs.tsukuba.ac.jp/index-j.html}{プログラム論理研究室} OB
      \end{itemize}
    \end{column}
  \end{columns}
\end{frame}

\section{``Mental Jinro''とは?}
\begin{frame}[fragile]
  \frametitle{``Mental Jinro''とは?}
  
   \uncover<2->{
    \begin{block}{}
      \begin{shadequote}[r]{}
        \textbf{Mental Jinro}は\textbf{人狼}から\textbf{ゲームマスター}を排除したゲームである。
      \end{shadequote}
    \end{block}
  }

  \begin{center}
    \uncover<3->{
      \begin{tikzpicture}
        \calloutquote[author={},width=6cm,position={(-2,-0.5)},fill=green!30,rounded corners]{\textbf{ゲームマスター}とは何か?}
      \end{tikzpicture}
    }

    \uncover<4->{
      \begin{tikzpicture}
        \calloutquote[author={},width=6cm,position={(3,-0.5)},fill=blue!30,rounded corners]{そもそも\textbf{人狼}とは何か?}
      \end{tikzpicture}
    }
  \end{center}
\end{frame}

\subsection{人狼とは?}
\begin{frame}[fragile]
  \frametitle{人狼とは?}

  \begin{exampleblock}{人狼\footnote[frame]{人狼には様々なルールがあるが、このスライドではこのルールを用いる}}
    \begin{shadequote}[r]{\scriptsize \href{https://ja.wikipedia.org/wiki/\%E6\%B1\%9D\%E3\%81\%AF\%E4\%BA\%BA\%E7\%8B\%BC\%E3\%81\%AA\%E3\%82\%8A\%E3\%82\%84\%3F}{Wikipedia ------ 汝は人狼なりや?}}
      \begin{itemize}
        \item<2-> プレイヤーはそれぞれが\textbf{村人}と村人に化けた\textbf{人狼}となり、自分自身の正体がばれないように他のプレイヤーと交渉して正体を探る
        \item<3-> ゲームは半日単位で進行し、昼には全プレイヤーの投票により決まった人狼容疑者の\textbf{処刑}が、夜には人狼による村人の\textbf{襲撃}が行われる
        \item
          \begin{itemize}
            \item<4-> 全ての人狼を処刑することができれば村人チームの勝ち
            \item<5-> 人狼と同じ数まで村人を減らすことができれば人狼チームの勝ち
          \end{itemize}
      \end{itemize}
    \end{shadequote}
  \end{exampleblock}
\end{frame}

\subsection{人狼の役職}
\begin{frame}[fragile]
  \frametitle{人狼の役職}

  \begin{block}{人狼に必要な役職}
    参加者を次の役職に分ける必要がある
    \begin{itemize}
      \item<2-> 村人
      \item<3-> 人狼
      \item<4-> \textbf{ゲームマスター}
    \end{itemize}
  \end{block}

  \begin{center}
    \uncover<5->{
      \begin{tikzpicture}
        \calloutquote[author={},width=6cm,position={(-2,-0.5)},fill=green!30,rounded corners]{\textbf{ゲームマスター}とは何か?}
      \end{tikzpicture}
    }
  \end{center}
\end{frame}

\subsection{ゲームマスターと公平性}
\begin{frame}[fragile]
  \frametitle{ゲームマスターと公平性}

  \begin{block}{ゲームマスターの役割}
    \begin{itemize}
      \item<2-> 人狼と村人のチーム分けをする
      \item<3-> 人狼に襲撃された村人を村人チームに宣告する
      \item<4-> 人狼と村人の数を管理し、どちらかのチームが勝利した時それを宣言する
    \end{itemize}
  \end{block}

  \begin{center}
    \uncover<5->{
      \begin{tikzpicture}
        \calloutquote[author={},width=5cm,position={(-2,-0.5)},fill=green!30,rounded corners]{\textbf{ゲームマスター}とは審判}
      \end{tikzpicture}
    }

    \uncover<6->{
      \begin{tikzpicture}
        \calloutquote[author={},width=7cm,position={(3,-0.5)},fill=blue!30,rounded corners]{ゲームマスターが\textbf{不公平}だったら?}
      \end{tikzpicture}
    }

    \uncover<7->{
      \begin{tikzpicture}
        \calloutquote[author={},width=2cm,position={(-2,-0.5)},fill=red!30,rounded corners]{\textbf{大問題!}}
      \end{tikzpicture}
    }
  \end{center}
\end{frame}

\subsection{Mental Jinro}
\begin{frame}[fragile]
  \frametitle{Mental Jinro}

  \begin{center}
    \begin{tikzpicture}
      \calloutquote[author={\includegraphics[width=1cm]{./img/bird2x.png}},width=10cm,position={(-1.5,-0.5)},fill=green!30,rounded corners]{\huge\textbf{ゲームマスター}を消そう!}
    \end{tikzpicture}
  \end{center}
\end{frame}

\begin{frame}{目次}
  \tableofcontents
\end{frame}

\end{document}
