\definecolor{links}{HTML}{2A1B81}
\hypersetup{colorlinks,linkcolor=,urlcolor=links}

\usetheme{Boadilla}
\usecolortheme{seahorse}
\usefonttheme{serif}
\beamertemplatenavigationsymbolsempty

\usepackage{luacode}
\usepackage{luatexja}
\usepackage{pgfpages}
\usepackage[osf]{mathpazo}

\begin{luacode*}
  USE_IPAFONT = os.getenv"USE_IPAFONT"
  USE_YUFONT = os.getenv"USE_YUFONT"
  
  if USE_YUFONT == "true" then
    tex.sprint("\\AtBeginDocument{\\usepackage[yu-osx, deluxe, expert]{luatexja-preset}}")
    tex.sprint("\\AtBeginDocument{\\setmainjfont[BoldFont=YuGothic Bold]{YuGothic Medium}}")
  elseif USE_IPAFONT == "true" then
    tex.sprint("\\AtBeginDocument{\\usepackage[ipaex, deluxe, expert]{luatexja-preset}}")
  else
    tex.sprint("\\AtBeginDocument{\\usepackage[hiragino-pro, deluxe, expert]{luatexja-preset}}")
    tex.sprint("\\AtBeginDocument{\\setmainjfont[BoldFont=Hiragino Kaku Gothic Pro W6]{Hiragino Kaku Gothic Pro W3}}")
  end
\end{luacode*}

\usepackage{epigraph}
\usepackage{etoolbox}
\usepackage{tikz}
\usepackage{framed}
\usepackage{libertine}
\usepackage[final]{listings}
\usepackage{amsmath}
\usepackage{mathtools}

%\setbeameroption{show notes on second screen=right}

%\usetikzlibrary{arrows,automata,shapes,backgrounds}

\setmainfont[Numbers=OldStyle, BoldFont=Palatino Bold]{Palatino}
\setsansfont{CMU Sans Serif}
\setmonofont{CMU Typewriter Text}

\title[Mental Jinroを支える暗号技術]{%
  \href{https://github.com/y-yu/mental-jinro-slide}{Mental Jinroを支える暗号技術} \\
  \href{http://connpass.com/event/30661/}{\normalsize tsukuba.pm \#3}
}
\author{吉村 優}
\date{May 14, 2016}
\institute[\url{https://twitter.com/\_yyu\_}]{%
  \url{https://twitter.com/\_yyu\_}\\
  \url{http://qiita.com/yyu}\\
  \url{https://github.com/y-yu}\\
}

\input{./lib/quotebox.tex}
\input{./lib/listings.tex}
\input{./lib/footnotemark.tex}
\input{./lib/mydescription.tex}
\input{./lib/ballon.tex}

\newcommand\ballref[1]{%
\tikz \node[circle, shade,ball color=structure.fg,inner sep=0pt,%
  text width=8pt,font=\tiny,align=center] {\color{white}\ref{#1}};
}

%\everymath{\displaystyle}

\begin{document}

\frame{\maketitle}

\section{自己紹介}
\begin{frame}[fragile]
  \frametitle{自己紹介}
  
  \begin{columns}
    \begin{column}{0.4\textwidth}
      \centering
      \begin{figure}
        \includegraphics[width=0.9\textwidth]{img/bird2x.png}
      \end{figure}
    \end{column}
    \begin{column}{0.6\textwidth}
      \begin{itemize}
        \item<2-> \href{https://www.coins.tsukuba.ac.jp/}{筑波大学 情報科学類} 学士(COINS11)
        \item<3-> \href{http://www.word-ac.net/}{WORD編集部} OB
        \item<4-> \href{http://logic.cs.tsukuba.ac.jp/index-j.html}{プログラム論理研究室} OB
      \end{itemize}
    \end{column}
  \end{columns}
\end{frame}

\section{Mental Jinroとは?}
\begin{frame}[fragile]
  \frametitle{Mental Jinroとは?}
  
   \uncover<2->{
    \begin{block}{}
      \begin{shadequote}[r]{}
        \emph{Mental Jinro}は\textbf{人狼}から\textbf{ゲームマスター}を排除したゲームである
      \end{shadequote}
    \end{block}
  }

  \begin{center}
    \uncover<3->{
      \begin{tikzpicture}
        \calloutquote[author={},width=6cm,position={(-1.2,-0.5)},fill=green!30,rounded corners]{\textbf{ゲームマスター}とは何か?}
      \end{tikzpicture}
    }

    \uncover<4->{
      \begin{tikzpicture}
        \calloutquote[author={},width=6cm,position={(1,-0.5)},fill=blue!30,rounded corners]{そもそも\textbf{人狼}とは何か?}
      \end{tikzpicture}
    }
  \end{center}
\end{frame}

\subsection{人狼とは?}
\begin{frame}[fragile]
  \frametitle{人狼とは?}

  \begin{exampleblock}{人狼\footnote[frame]{人狼には様々なルールがあるが、このスライドではこのルールを用いる}}
    \begin{shadequote}[r]{\scriptsize \href{https://ja.wikipedia.org/wiki/\%E6\%B1\%9D\%E3\%81\%AF\%E4\%BA\%BA\%E7\%8B\%BC\%E3\%81\%AA\%E3\%82\%8A\%E3\%82\%84\%3F}{Wikipedia ------ 汝は人狼なりや?}}
      \begin{itemize}
        \item<2-> プレイヤーはそれぞれが\textbf{村人}と村人に化けた\textbf{人狼}となり、自分自身の正体がばれないように他のプレイヤーと交渉して正体を探る
        \item<3-> ゲームは半日単位で進行し、昼には全プレイヤーの投票により決まった人狼容疑者の\textbf{処刑}が、夜には人狼による村人の\textbf{襲撃}が行われる
        \item
          \begin{itemize}
            \item<4-> 全ての人狼を処刑することができれば村人チームの勝ち
            \item<5-> 人狼と同じ数まで村人を減らすことができれば人狼チームの勝ち
          \end{itemize}
      \end{itemize}
    \end{shadequote}
  \end{exampleblock}
\end{frame}

\subsection{人狼の役職}
\begin{frame}[fragile]
  \frametitle{人狼の役職}

  \begin{block}{人狼に必要な役職}
    参加者を次の役職に分ける必要がある
    \begin{itemize}
      \item<2-> 村人
      \item<3-> 人狼
      \item<4-> \textbf{ゲームマスター}
    \end{itemize}
  \end{block}

  \begin{center}
    \uncover<5->{
      \begin{tikzpicture}
        \calloutquote[author={},width=6cm,position={(-1,-0.5)},fill=green!30,rounded corners]{\textbf{ゲームマスター}とは何か?}
      \end{tikzpicture}
    }
  \end{center}
\end{frame}

\subsection{ゲームマスターと公平性}
\begin{frame}[fragile]
  \frametitle{ゲームマスターと公平性}

  \begin{block}{ゲームマスターの役割}
    \begin{itemize}
      \item<2-> 人狼と村人のチーム分けをする
      \item<3-> 人狼に襲撃された村人を村人チームに宣告する
      \item<4-> 人狼と村人の数を管理し、どちらかのチームが勝利した時それを宣言する
    \end{itemize}
  \end{block}

  \begin{center}
    \uncover<5->{
      \begin{tikzpicture}
        \calloutquote[author={},width=5cm,position={(-1.3,-0.5)},fill=green!30,rounded corners]{\textbf{ゲームマスター}とは審判}
      \end{tikzpicture}
    }

    \uncover<6->{
      \begin{tikzpicture}
        \calloutquote[author={},width=7cm,position={(1.2,-0.5)},fill=blue!30,rounded corners]{ゲームマスターが\textbf{不公平}だったら?}
      \end{tikzpicture}
    }

    \uncover<7->{
      \begin{tikzpicture}
        \calloutquote[author={},width=2cm,position={(-1,-0.5)},fill=red!30,rounded corners]{\textbf{大問題!}}
      \end{tikzpicture}
    }
  \end{center}
\end{frame}

\subsection{Mental Jinro}
\begin{frame}[fragile]
  \frametitle{Mental Jinro}

  \begin{center}
    \begin{tikzpicture}
      \calloutquote[author={\includegraphics[width=1cm]{./img/bird2x.png}},width=11cm,position={(-1.1,-0.5)},fill=green!30,rounded corners, execute at begin node=\huge]{\textbf{ゲームマスター}を消そう!}
    \end{tikzpicture}
  \end{center}

  \uncover<2->{
    \begin{alertblock}{ゲームマスターが消えると……}
      \begin{itemize}
        \item<3-> チーム分けはどうする?
        \item<4-> 襲撃された村人の情報をどう伝える?
        \item<5-> 勝敗は誰が判断する? 
      \end{itemize}
    \end{alertblock}
  }
\end{frame}

\begin{frame}
  \centering
  {\huge Mental Jinroを支える暗号技術 \\ \uncover<2->{\Huge\textbf{コミットメント}}}
\end{frame}

\section{コミットメント}

\subsection{コイントスゲーム}
\begin{frame}[fragile]
  \frametitle{コイントスゲーム}

  \begin{exampleblock}{}
    アリスとボブの二人がいるとする
    \begin{enumerate}
      \item<2-> アリスがコインの``表''または``裏''を紙に書き、紙を\textbf{封筒}に入れる
      \item<3-> ボブはコインを投げる
      \item<4-> 封筒から紙を取り出し、
        \begin{itemize}
          \item アリスの予想とコインの結果が同じなら、アリスの勝利
          \item アリスの予想とコインの結果が違えば、ボブの勝利
        \end{itemize}
      \item<5-> このゲームは\textbf{電話上}で行う
    \end{enumerate}
  \end{exampleblock}

  \begin{center}
    \uncover<6->{
      \begin{tikzpicture}
        \calloutquote[author={},width=6cm,position={(-1.2,-0.3)},fill=green!30,rounded corners]{アリスが予想を\textbf{反故}にする?}
      \end{tikzpicture}
    }

    \uncover<7->{
      \begin{tikzpicture}
        \calloutquote[author={},width=6.5cm,position={(1.1,-0.3)},fill=blue!30,rounded corners]{\textbf{封筒}をどうやって実現する?}
      \end{tikzpicture}
    }
  \end{center}
\end{frame}

\begin{frame}[fragile, label=protocol]
  \frametitle{コイントスゲーム}

  \begin{block}{プロトコル}
    \begin{enumerate}
      \item<2-> ボブは$p = 2q + 1$となる大きな素数$p, q$をランダムに生成して、
        $\mathbb{Z}^*_p$\footnote[frame]{整数$x \bmod p$かつ$xy \equiv 1 \pmod{p}$となる逆元$y$が存在する$x$の集合である}
        の位数$q$の部分群$G$から\textbf{生成元}\footnote[frame]{後述する}$g, v$をランダムに選択して$p, q, g, v$をアリスへ送信する
      \item<3-> アリスは$p, q, g, v$を検証し、表と予想するなら\textcolor{red}{$m := 1$}を選択し、
        裏と予想するなら\textcolor{blue}{$m := q - 1$}を選択し、
        乱数$r \in \{1, \dots, q - 1\}$を用いて$c := g^r v^m \bmod p$計算し$c$をボブへ送信する
      \item<4-> ボブはコイントスをして、結果をアリスへ送信する
      \item<5-> アリスは$r, m$を公開する
      \item<6-> ボブは$c \equiv g^r v^m \pmod{p}$を検証する
    \end{enumerate}
  \end{block}
\end{frame}

\subsection{コイントスゲームの検証}
\begin{frame}[fragile, label=mdash]
  \frametitle{コイントスゲームの検証}
  
  \begin{center}
    \uncover<2->{
      \begin{tikzpicture}
        \calloutquote[author={},width=6cm,position={(-1.2,-0.3)},fill=green!30,rounded corners]{アリスが予想を\textbf{反故}にする?}
      \end{tikzpicture}
    }

    \uncover<3->{
      アリスは$m$をコミットした後で、$m'(m' \ne m)$と偽れる

      {\Large$\Downarrow$\scriptsize ならば}
    }

    \uncover<4->{
      アリスは$g^r v^m = g^{r'} v^{m'}$となる$r'$を計算できる

      {\Large$\Downarrow$\scriptsize ならば}
    }

    \uncover<5->{
      アリスは$g$を何乗したら$v$となるかという\textbf{離散対数}が求められる

      \begin{align*}
        g^r v^m    &\equiv g^{r'} v^{m'}      &\pmod{p} \\
        v^{m - m'} &\equiv g^{r' - r}         &\pmod{p} \\
        \log_g{(v^{m - m'})} &\equiv r' - r   &\pmod{q} \\
        \log_g{v} &\equiv (r' - r) / (m - m') &\pmod{q}
      \end{align*}
    }
  \end{center}
\end{frame}

\subsection{離散対数問題}
\begin{frame}[fragile]
  \frametitle{離散対数問題}

  \begin{exampleblock}{離散対数問題}
    \begin{shadequote}[r]{\scriptsize クラウドを支えるこれからの暗号技術\cite{光成201506}}
      次の条件を満す$g, p, y(y = g^x \bmod p)$が与えられたとき、$x$を求める問題のことである
    \end{shadequote}
  \end{exampleblock}

  \uncover<2->{
    \begin{block}{}
      $g, p$が次を満すとき、離散対数問題を解くことは困難
      \begin{itemize}
        \item<3-> $p$は巨大な素数
        \item<4-> $p - 1$の約数の中に、巨大な素数$q$が含まれている
        \item<5-> $g$は全ての$i = 1, \dots, q - 1$について、$g^i \not\equiv 1 \pmod{p}$となる%
          \footnote[frame]{このような$g$のことを\textbf{生成元}と言う}
      \end{itemize}
    \end{block}
  }
\end{frame}

\againframe<6>{protocol}
\againframe<5>{mdash}

\begin{frame}[fragile]
  \frametitle{コイントスゲームの検証}

  \begin{center}
    アリスは$m$をコミットした後で、$m'(m' \ne m)$と偽れる

    {\Large$\Downarrow$\scriptsize ならば}

    アリスは$g$を何乗したら$v$となるかという\textbf{離散対数}が求められる
  \end{center}

  \uncover<2->{
    \begin{alertblock}{}
      \begin{center}
        離散対数問題を解くことは困難であるということに矛盾する
      \end{center}
    \end{alertblock}
  }

  \uncover<3->{
    \begin{block}{}
      \begin{center}
        アリスは$m$をコミットした後で$m'(m \ne m)$と偽ることは困難である
      \end{center}
    \end{block}
  }
\end{frame}

\begin{frame}[fragile]
  \frametitle{コイントスゲームの検証}

  \begin{center}
    \begin{tikzpicture}
      \calloutquote[author={},width=6.5cm,position={(1.1,-0.3)},fill=blue!30,rounded corners]{\textbf{封筒}をどうやって実現する?}
    \end{tikzpicture}

    \uncover<2->{
      ボブは$g^r v^m$から$m$を特定できる

      {\Large$\downarrow$\scriptsize しかし}
    }

    \uncover<3->{
      $g, v$は\textbf{生成元}である

      {\Large$\Downarrow$\scriptsize なので}
      
      $g^r \bmod p$と$v^m \bmod p$は$1, \dots, p - 1$の全ての値を取る

      {\Large$\Downarrow$\scriptsize つまり}
    }

    \uncover<4->{
      全ての$m'$には$g^r v^m = g^{r'} v^{m'}$となる$r'$が存在する

      {\Large$\Downarrow$\scriptsize つまり}
    }

    \uncover<5->{
      \begin{alertblock}{}
        \begin{center}
          ボブは正しい$m$を\textbf{区別}することができない
        \end{center}
      \end{alertblock}
    }
  \end{center}
\end{frame}

\section{まとめ}
\begin{frame}[fragile]
  \frametitle{まとめ}

  \begin{itemize}
    \item<2-> Mental Jinroはゲームマスターを排除した人狼である
    \item<3-> コミットメントを利用することで、コミットした情報を反故にしたり、
      コミットメントからコミットした情報を特定されることを防げる
    \item<4-> Mental Jinroはこのコミットメントによって成り立っている
    \item<5-> Mental Jinroの詳細は\href{http://qiita.com/yyu/items/8c10fcdbc17084ac2674}{Qiitaの記事}を参照のこと
  \end{itemize}
\end{frame}

\begin{frame}{目次}
  \tableofcontents
\end{frame}

\section*{参考文献}
\begin{frame}
  \frametitle{参考文献}

  \nocite{*}
  \bibliographystyle{jplain_url}
  \bibliography{ref}
\end{frame}

\begin{frame}
  \centering
  {\Huge Thank you for listening!}
  
  {\Huge Any question?}
\end{frame}

\end{document}
