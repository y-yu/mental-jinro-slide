\definecolor{links}{HTML}{2A1B81}
\hypersetup{colorlinks,linkcolor=,urlcolor=links}

\usetheme{Boadilla}
\usecolortheme{seahorse}
\usefonttheme{serif}
\beamertemplatenavigationsymbolsempty

\usepackage{luacode}
\usepackage{luatexja}
\usepackage{pgfpages}
\usepackage[osf]{mathpazo}
\usepackage{fontspec}

\begin{luacode*}
  USE_IPAFONT = os.getenv"USE_IPAFONT"
  
  if USE_IPAFONT == "true" then
    tex.sprint("\\AtBeginDocument{\\usepackage[ipaex, deluxe, expert]{luatexja-preset}}")
  else
    tex.sprint("\\AtBeginDocument{\\usepackage[hiragino-pro, deluxe, expert]{luatexja-preset}}")
    tex.sprint("\\AtBeginDocument{\\setmainjfont[BoldFont=Hiragino Kaku Gothic Pro W6]{Hiragino Kaku Gothic Pro W3}}")
  end
\end{luacode*}

\usepackage{epigraph}
\usepackage{etoolbox}
\usepackage{tikz}
\usepackage{framed}
\usepackage{libertine}
\usepackage[final]{listings}
\usepackage{amsmath}
\usepackage{mathtools}

%\setbeameroption{show notes on second screen=right}

%\usetikzlibrary{arrows,automata,shapes,backgrounds}

\setmainfont[Numbers=OldStyle, BoldFont=Palatino Bold]{Palatino}
\setsansfont{CMU Sans Serif}
\setmonofont{CMU Typewriter Text}

\title[Mental Jinro]{%
  \href{https://github.com/y-yu/mental-jinro-slide}{Mental Jinro} \\
  \href{http://connpass.com/event/30661/}{\normalsize tsukuba.pm \#3}
}
\author{吉村 優}
\date{May 14, 2016}
\institute[\url{https://twitter.com/\_yyu\_}]{%
  \url{https://twitter.com/\_yyu\_}\\
  \url{http://qiita.com/yyu}\\
  \url{https://github.com/y-yu}\\
}

\input{./lib/quotebox.tex}
\input{./lib/listings.tex}
\input{./lib/footnotemark.tex}
\input{./lib/mydescription.tex}

\newcommand\ballref[1]{%
\tikz \node[circle, shade,ball color=structure.fg,inner sep=0pt,%
  text width=8pt,font=\tiny,align=center] {\color{white}\ref{#1}};
}

%\everymath{\displaystyle}

\begin{document}

\frame{\maketitle}

\section{自己紹介}
\begin{frame}[fragile]
  \frametitle{自己紹介}
  
  \begin{columns}
    \begin{column}{0.25\textwidth}
      \centering
      \begin{figure}
        \includegraphics[width=\textwidth]{img/bird2x.png}
      \end{figure}
    \end{column}
    \begin{column}{0.75\textwidth}
      \begin{itemize}
        \item<2-> 吉村 優
        \item<3-> COINS11
        \item<4-> WORD編集部OB
      \end{itemize}
    \end{column}
  \end{columns}
\end{frame}

\section{``Mental Jinro''とは}
\begin{frame}[fragile]
  \frametitle{``Mental Jinro''とは}
  
   \uncover<2->{
    \begin{block}{}
      \begin{shadequote}[r]{}
        \textbf{Mental Jinro}は\textbf{人狼}からゲームマスター(審判)を排除したゲームである。
      \end{shadequote}
    \end{block}
  }

  \uncover<3->{
    \begin{alertblock}{}
      \centering
      そもそも\textbf{人狼}とはどのようなゲームか?
    \end{alertblock}
  }
\end{frame}

\subsection{人狼の説明}
\begin{frame}[fragile]
  \frametitle{``Mental Jinro''とは}

\end{frame}

\end{document}
